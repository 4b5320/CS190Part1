 \section{Proposed Algorithm: BRCGA + LS}
    Using the same smaller scale version of the problem, the proposed algorithm discussed in Chapter 3 is used to also find the optimal locations for the two wind turbines in a heuristical manner. The parameters shown in Table \ref{parametersGA} were used.
    
    \begin{table}[H]
        \centering
        \begin{tabular}{|c|c|} \hline
            \textbf{Population size} &100 \\ \hline
            \textbf{Maximum generation} &1000 \\ \hline
            \textbf{Mutation rate} &0.2 \\ \hline
            \textbf{Crossover rate} &0.7 \\
            \hline
        \end{tabular}
        \caption{Parameters used for the proposed algorithm on the small scale problem}
        \label{parametersGA}
    \end{table}
    
    Since the problem is in small scale, the algorithm only reached 15th generation with the same solution as with the exhaustive search. Moreover, the wind farm with two wind turbines has a cost of
    \begin{align*}
        Cost_{tot}
        &= N\left(\frac{2}{3} + \frac{1}{3}e^{-0.00174N^2}\right) \\
        &= \left(2\right)\left(\frac{2}{3} + \frac{1}{3}e^{-0.00174\left(2^2\right)}\right) \\
        &= 1.995
    \end{align*}
    Hence, the fitness of the chromosome representing the optimal configuration of the wind farm with two wind turbines is
    \begin{align*}
        obj
        &=\frac{Cost_{tot}}{P_{tot}} \\
        &=\frac{1.995}{1027.0990kW} \\
        &=0.001942kW^{-1}
    \end{align*}